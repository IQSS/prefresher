\documentclass[]{book}
\usepackage{lmodern}
\usepackage{amssymb,amsmath}
\usepackage{ifxetex,ifluatex}
\usepackage{fixltx2e} % provides \textsubscript
\ifnum 0\ifxetex 1\fi\ifluatex 1\fi=0 % if pdftex
  \usepackage[T1]{fontenc}
  \usepackage[utf8]{inputenc}
\else % if luatex or xelatex
  \ifxetex
    \usepackage{mathspec}
  \else
    \usepackage{fontspec}
  \fi
  \defaultfontfeatures{Ligatures=TeX,Scale=MatchLowercase}
\fi
% use upquote if available, for straight quotes in verbatim environments
\IfFileExists{upquote.sty}{\usepackage{upquote}}{}
% use microtype if available
\IfFileExists{microtype.sty}{%
\usepackage{microtype}
\UseMicrotypeSet[protrusion]{basicmath} % disable protrusion for tt fonts
}{}
\usepackage[margin=1in]{geometry}
\usepackage{hyperref}
\hypersetup{unicode=true,
            pdftitle={A Minimal Book Example},
            pdfauthor={Yihui Xie},
            pdfborder={0 0 0},
            breaklinks=true}
\urlstyle{same}  % don't use monospace font for urls
\usepackage{natbib}
\bibliographystyle{apalike}
\usepackage{color}
\usepackage{fancyvrb}
\newcommand{\VerbBar}{|}
\newcommand{\VERB}{\Verb[commandchars=\\\{\}]}
\DefineVerbatimEnvironment{Highlighting}{Verbatim}{commandchars=\\\{\}}
% Add ',fontsize=\small' for more characters per line
\usepackage{framed}
\definecolor{shadecolor}{RGB}{248,248,248}
\newenvironment{Shaded}{\begin{snugshade}}{\end{snugshade}}
\newcommand{\KeywordTok}[1]{\textcolor[rgb]{0.13,0.29,0.53}{\textbf{#1}}}
\newcommand{\DataTypeTok}[1]{\textcolor[rgb]{0.13,0.29,0.53}{#1}}
\newcommand{\DecValTok}[1]{\textcolor[rgb]{0.00,0.00,0.81}{#1}}
\newcommand{\BaseNTok}[1]{\textcolor[rgb]{0.00,0.00,0.81}{#1}}
\newcommand{\FloatTok}[1]{\textcolor[rgb]{0.00,0.00,0.81}{#1}}
\newcommand{\ConstantTok}[1]{\textcolor[rgb]{0.00,0.00,0.00}{#1}}
\newcommand{\CharTok}[1]{\textcolor[rgb]{0.31,0.60,0.02}{#1}}
\newcommand{\SpecialCharTok}[1]{\textcolor[rgb]{0.00,0.00,0.00}{#1}}
\newcommand{\StringTok}[1]{\textcolor[rgb]{0.31,0.60,0.02}{#1}}
\newcommand{\VerbatimStringTok}[1]{\textcolor[rgb]{0.31,0.60,0.02}{#1}}
\newcommand{\SpecialStringTok}[1]{\textcolor[rgb]{0.31,0.60,0.02}{#1}}
\newcommand{\ImportTok}[1]{#1}
\newcommand{\CommentTok}[1]{\textcolor[rgb]{0.56,0.35,0.01}{\textit{#1}}}
\newcommand{\DocumentationTok}[1]{\textcolor[rgb]{0.56,0.35,0.01}{\textbf{\textit{#1}}}}
\newcommand{\AnnotationTok}[1]{\textcolor[rgb]{0.56,0.35,0.01}{\textbf{\textit{#1}}}}
\newcommand{\CommentVarTok}[1]{\textcolor[rgb]{0.56,0.35,0.01}{\textbf{\textit{#1}}}}
\newcommand{\OtherTok}[1]{\textcolor[rgb]{0.56,0.35,0.01}{#1}}
\newcommand{\FunctionTok}[1]{\textcolor[rgb]{0.00,0.00,0.00}{#1}}
\newcommand{\VariableTok}[1]{\textcolor[rgb]{0.00,0.00,0.00}{#1}}
\newcommand{\ControlFlowTok}[1]{\textcolor[rgb]{0.13,0.29,0.53}{\textbf{#1}}}
\newcommand{\OperatorTok}[1]{\textcolor[rgb]{0.81,0.36,0.00}{\textbf{#1}}}
\newcommand{\BuiltInTok}[1]{#1}
\newcommand{\ExtensionTok}[1]{#1}
\newcommand{\PreprocessorTok}[1]{\textcolor[rgb]{0.56,0.35,0.01}{\textit{#1}}}
\newcommand{\AttributeTok}[1]{\textcolor[rgb]{0.77,0.63,0.00}{#1}}
\newcommand{\RegionMarkerTok}[1]{#1}
\newcommand{\InformationTok}[1]{\textcolor[rgb]{0.56,0.35,0.01}{\textbf{\textit{#1}}}}
\newcommand{\WarningTok}[1]{\textcolor[rgb]{0.56,0.35,0.01}{\textbf{\textit{#1}}}}
\newcommand{\AlertTok}[1]{\textcolor[rgb]{0.94,0.16,0.16}{#1}}
\newcommand{\ErrorTok}[1]{\textcolor[rgb]{0.64,0.00,0.00}{\textbf{#1}}}
\newcommand{\NormalTok}[1]{#1}
\usepackage{longtable,booktabs}
\usepackage{graphicx,grffile}
\makeatletter
\def\maxwidth{\ifdim\Gin@nat@width>\linewidth\linewidth\else\Gin@nat@width\fi}
\def\maxheight{\ifdim\Gin@nat@height>\textheight\textheight\else\Gin@nat@height\fi}
\makeatother
% Scale images if necessary, so that they will not overflow the page
% margins by default, and it is still possible to overwrite the defaults
% using explicit options in \includegraphics[width, height, ...]{}
\setkeys{Gin}{width=\maxwidth,height=\maxheight,keepaspectratio}
\IfFileExists{parskip.sty}{%
\usepackage{parskip}
}{% else
\setlength{\parindent}{0pt}
\setlength{\parskip}{6pt plus 2pt minus 1pt}
}
\setlength{\emergencystretch}{3em}  % prevent overfull lines
\providecommand{\tightlist}{%
  \setlength{\itemsep}{0pt}\setlength{\parskip}{0pt}}
\setcounter{secnumdepth}{5}
% Redefines (sub)paragraphs to behave more like sections
\ifx\paragraph\undefined\else
\let\oldparagraph\paragraph
\renewcommand{\paragraph}[1]{\oldparagraph{#1}\mbox{}}
\fi
\ifx\subparagraph\undefined\else
\let\oldsubparagraph\subparagraph
\renewcommand{\subparagraph}[1]{\oldsubparagraph{#1}\mbox{}}
\fi

%%% Use protect on footnotes to avoid problems with footnotes in titles
\let\rmarkdownfootnote\footnote%
\def\footnote{\protect\rmarkdownfootnote}

%%% Change title format to be more compact
\usepackage{titling}

% Create subtitle command for use in maketitle
\newcommand{\subtitle}[1]{
  \posttitle{
    \begin{center}\large#1\end{center}
    }
}

\setlength{\droptitle}{-2em}
  \title{A Minimal Book Example}
  \pretitle{\vspace{\droptitle}\centering\huge}
  \posttitle{\par}
  \author{Yihui Xie}
  \preauthor{\centering\large\emph}
  \postauthor{\par}
  \predate{\centering\large\emph}
  \postdate{\par}
  \date{2018-06-16}

\usepackage{booktabs}

\usepackage{epsfig}
\usepackage{epstopdf}
\usepackage{rotate}
\usepackage{graphicx}
\usepackage{hyperref}
\usepackage{alphalph}
\usepackage{caption}
\usepackage[hang,flushmargin]{footmisc}
\usepackage{framed}
\usepackage{xcolor}
\usepackage{verbatim} 

\setcounter{MaxMatrixCols}{20}
\newcommand{\Var}{\mathrm{Var}}
\newcommand{\SD}{\mathrm{SD}}
\newcommand{\Cov}{\mathrm{Cov}}
\newcommand{\fx}{f({\bf x})}
\newcommand\R{{\textsf R~}}
\newcommand\Rst{\textsf{RStudio}}

\oddsidemargin=0in
\evensidemargin=0in
\textwidth=6.5in
\topmargin=0in
\headsep=.25in
\headheight=.25in
\textheight=9.25in
\topskip=0in
\voffset=-0.5in
\epsfxsize=1in

\usepackage{amsthm}
\newtheorem{theorem}{Theorem}[chapter]
\newtheorem{lemma}{Lemma}[chapter]
\theoremstyle{definition}
\newtheorem{definition}{Definition}[chapter]
\newtheorem{corollary}{Corollary}[chapter]
\newtheorem{proposition}{Proposition}[chapter]
\theoremstyle{definition}
\newtheorem{example}{Example}[chapter]
\theoremstyle{definition}
\newtheorem{exercise}{Exercise}[chapter]
\theoremstyle{remark}
\newtheorem*{remark}{Remark}
\newtheorem*{solution}{Solution}
\begin{document}
\maketitle

{
\setcounter{tocdepth}{1}
\tableofcontents
}
\chapter{Prerequisites}\label{prerequisites}

This is a \emph{sample} book written in \textbf{Markdown}. You can use
anything that Pandoc's Markdown supports, e.g., a math equation
\(a^2 + b^2 = c^2\).

The \textbf{bookdown} package can be installed from CRAN or Github:

\begin{Shaded}
\begin{Highlighting}[]
\KeywordTok{install.packages}\NormalTok{(}\StringTok{"bookdown"}\NormalTok{)}
\CommentTok{# or the development version}
\CommentTok{# devtools::install_github("rstudio/bookdown")}
\end{Highlighting}
\end{Shaded}

Remember each Rmd file contains one and only one chapter, and a chapter
is defined by the first-level heading \texttt{\#}.

To compile this example to PDF, you need XeLaTeX. You are recommended to
install TinyTeX (which includes XeLaTeX):
\url{https://yihui.name/tinytex/}.

\chapter{Introduction}\label{intro}

You can label chapter and section titles using \texttt{\{\#label\}}
after them, e.g., we can reference Chapter \ref{intro}. If you do not
manually label them, there will be automatic labels anyway, e.g.,
Chapter \ref{methods}.

Figures and tables with captions will be placed in \texttt{figure} and
\texttt{table} environments, respectively.

\begin{Shaded}
\begin{Highlighting}[]
\KeywordTok{par}\NormalTok{(}\DataTypeTok{mar =} \KeywordTok{c}\NormalTok{(}\DecValTok{4}\NormalTok{, }\DecValTok{4}\NormalTok{, .}\DecValTok{1}\NormalTok{, .}\DecValTok{1}\NormalTok{))}
\KeywordTok{plot}\NormalTok{(pressure, }\DataTypeTok{type =} \StringTok{'b'}\NormalTok{, }\DataTypeTok{pch =} \DecValTok{19}\NormalTok{)}
\end{Highlighting}
\end{Shaded}

\begin{figure}

{\centering \includegraphics[width=0.8\linewidth]{prefresher_files/figure-latex/nice-fig-1} 

}

\caption{Here is a nice figure!}\label{fig:nice-fig}
\end{figure}

Reference a figure by its code chunk label with the \texttt{fig:}
prefix, e.g., see Figure \ref{fig:nice-fig}. Similarly, you can
reference tables generated from \texttt{knitr::kable()}, e.g., see Table
\ref{tab:nice-tab}.

\begin{Shaded}
\begin{Highlighting}[]
\NormalTok{knitr}\OperatorTok{::}\KeywordTok{kable}\NormalTok{(}
  \KeywordTok{head}\NormalTok{(iris, }\DecValTok{20}\NormalTok{), }\DataTypeTok{caption =} \StringTok{'Here is a nice table!'}\NormalTok{,}
  \DataTypeTok{booktabs =} \OtherTok{TRUE}
\NormalTok{)}
\end{Highlighting}
\end{Shaded}

\begin{table}

\caption{\label{tab:nice-tab}Here is a nice table!}
\centering
\begin{tabular}[t]{rrrrl}
\toprule
Sepal.Length & Sepal.Width & Petal.Length & Petal.Width & Species\\
\midrule
5.1 & 3.5 & 1.4 & 0.2 & setosa\\
4.9 & 3.0 & 1.4 & 0.2 & setosa\\
4.7 & 3.2 & 1.3 & 0.2 & setosa\\
4.6 & 3.1 & 1.5 & 0.2 & setosa\\
5.0 & 3.6 & 1.4 & 0.2 & setosa\\
\addlinespace
5.4 & 3.9 & 1.7 & 0.4 & setosa\\
4.6 & 3.4 & 1.4 & 0.3 & setosa\\
5.0 & 3.4 & 1.5 & 0.2 & setosa\\
4.4 & 2.9 & 1.4 & 0.2 & setosa\\
4.9 & 3.1 & 1.5 & 0.1 & setosa\\
\addlinespace
5.4 & 3.7 & 1.5 & 0.2 & setosa\\
4.8 & 3.4 & 1.6 & 0.2 & setosa\\
4.8 & 3.0 & 1.4 & 0.1 & setosa\\
4.3 & 3.0 & 1.1 & 0.1 & setosa\\
5.8 & 4.0 & 1.2 & 0.2 & setosa\\
\addlinespace
5.7 & 4.4 & 1.5 & 0.4 & setosa\\
5.4 & 3.9 & 1.3 & 0.4 & setosa\\
5.1 & 3.5 & 1.4 & 0.3 & setosa\\
5.7 & 3.8 & 1.7 & 0.3 & setosa\\
5.1 & 3.8 & 1.5 & 0.3 & setosa\\
\bottomrule
\end{tabular}
\end{table}

You can write citations, too. For example, we are using the
\textbf{bookdown} package \citep{R-bookdown} in this sample book, which
was built on top of R Markdown and \textbf{knitr} \citep{xie2015}.

\chapter{Functions and Notation}\label{functions-and-notation}

Topics\footnote{Much of the material and examples for this lecture are taken from Simon \& Blume (1994) \emph{Mathematics for Economists}, Boyce \& Diprima (1988) \emph{Calculus}, and Protter \& Morrey (1991) \emph{A First Course in Real Analysis}}:\textbackslash{}

Dimensionality; Interval Notation for \({\bf R}^1\); Neighborhoods:
Intervals, Disks, and Balls; Introduction to Functions; Domain and
Range; Some General Types of Functions; Log, Ln, and e; Other Useful
Functions; Graphing Functions; Solving for Variables; Finding Roots;
Limit of a Function; Continuity; Sets, Sets, and More Sets.

\section{Dimensionality}\label{dimensionality}

\begin{itemize}
\tightlist
\item
  \({\bf R}^1\) is the set of all real numbers extending from
  \(-\infty\) to \(+\infty\) --- i.e., the real number line.
\item
  \({\bf R}^n\) is an \(n\)-dimensional space (often referred to as
  Euclidean space), where each of the \(n\) axes extends from
  \(-\infty\) to \(+\infty\).

  \begin{itemize}
  \tightlist
  \item
    \({\bf R}^1\) is a one dimensional line. \({\bf R}^2\) is a two
    dimensional plane. \({\bf R}^3\) is a three dimensional space.
    \({\bf R}^4\) could be 3-D plus time (or temperature, etc).
  \end{itemize}
\item
  Points in \({\bf R}^n\) are ordered \(n\)-tuples, where each element
  of the \(n\)-tuple represents the coordinate along that dimension.

  \begin{itemize}
  \tightlist
  \item
    \({\bf R}^1\): (3)
  \item
    \({\bf R}^2\): (-15, 5)
  \item
    \({\bf R}^3\): (86, 4, 0)\textbackslash{}
  \end{itemize}
\end{itemize}

\section{\texorpdfstring{Interval Notation for
\({\bf R}^1\)}{Interval Notation for \{\textbackslash{}bf R\}\^{}1}}\label{interval-notation-for-bf-r1}

Open interval: \[(a,b)\equiv \{ x\in{\bf R}^1: a<x<b\}\]

\(x\) is a one-dimensional element in which x is greater than a and less
than b

Closed interval: \[[a,b]\equiv \{ x\in{\bf R}^1: a\le x \le b\}\]

\(x\) is a one-dimensional element in which x is greater or equal to
than a and less than or equal to b

Half open, half closed: \[(a,b]\equiv \{ x\in{\bf R}^1: a<x\le b\}\]

\(x\) is a one-dimensional element in which x is greater than a and less
than or equal to b

\section{Neighborhoods: Intervals, Disks, and
Balls}\label{neighborhoods-intervals-disks-and-balls}

In many areas of math, we need a formal construct for what it means to
be ``near'' a point \(\bf c\) in \({\bf R}^n\). This is generally called
the \textbf{neighborhood} of \(\bf c\). It's represented by an open
interval, disk, or ball, depending on whether \({\bf R}^n\) is of one,
two, or more dimensions, respectively.

Given the point \(c\), these are defined as

\begin{itemize}
\tightlist
\item
  \{\bf $\epsilon$-interval\} in \({\bf R}^1\):\}
  \(\{x : |x-c|<\epsilon \}\). \(x\) is in the neighborhood of \{\bf c\}
  if it is in the open interval \((c-\epsilon,c+\epsilon)\).
\item
  \{\bf $\epsilon$-disk\} in \({\bf R}^2\):
  \(\{x : || x-c ||<\epsilon\}\). \(x\) is in the neighborhood of
  \{\bf c\} if it is inside the circle or disc with center \(\bf c\) and
  radius \(\epsilon\).
\item
  \bf $\epsilon$-ball in ${\bf R}^n$:\} \(\{x : || x-c ||<\epsilon\}\).
  \(x\) is in the neighborhood of \{\bf c\} if it is inside the sphere
  or ball with center \(\bf c\) and radius \(\epsilon\).\textbackslash{}
\end{itemize}

\section{Introduction to Functions}\label{introduction-to-functions}

A \{\bf function\} (in \({\bf R}^1\)) is a mapping, or transformation,
that relates members of one set to members of another set. For instance,
if you have two sets: set \(A\) and set \(B\), a function from \(A\) to
\(B\) maps every value \(a\) in set \(A\) such that \(f(a) \in B\).
Functions can be
\texttt{many-to-one",\ where\ many\ values\ or\ combinations\ of\ values\ from\ set\ \$A\$\ produce\ a\ single\ output\ in\ set\ \$B\$,\ or\ they\ can\ be}one-to-one``,
where each value in set \(A\) corresponds to a single value in set
\(B\).

Examples: Mapping notation

\begin{itemize}
\tightlist
\item
  Function of one variable: \(f:{\bf R}^1\to{\bf R}^1\)\textbackslash{}
  \(f(x)=x+1\)\textbackslash{} For each \(x\) in \({\bf R}^1\), \(f(x)\)
  assigns the number \(x+1\).
\item
  Function of two variables:
  \(f: {\bf R}^2\to{\bf R}^1\)\textbackslash{}
  \(f(x,y)=x^2+y^2\)\textbackslash{} For each ordered pair \((x,y)\) in
  \({\bf R}^2\), \(f(x,y)\) assigns the number \(x^2+y^2\).
\end{itemize}

We often use variable \(x\) as input and another \(y\) as output, e.g.
\(y=x+1\)

\section{Domain and Range/Image}\label{domain-and-rangeimage}

Some functions are defined only on proper subsets of \({\bf R}^n\). *
\{\bf Domain\}: the set of numbers in \(X\) at which \(f(x)\) is
defined. * \{\bf Range\}: elements of \(Y\) assigned by \(f(x)\) to
elements of \(X\), or \[f(X)=\{ y : y=f(x), x\in X\}\] Most often used
when talking about a function \(f:{\bf R}^1\to{\bf R}^1\). *
\{\bf Image\}: same as range, but more often used when talking about a
function \(f:{\bf R}^n\to{\bf R}^1\).

\begin{verbatim}
<!-- \begin{samepage}    -->
<!-- \begin{framed} -->
<!-- \item[] Examples: -->
\end{verbatim}

--\textgreater{}

--\textgreater{}

--\textgreater{}

\section{Some General Types of
Functions}\label{some-general-types-of-functions}

\begin{verbatim}
\begin{itemize}
\item {\bf Monomials}:  $f(x)=a x^k$\\
    $a$ is the coefficient.  $k$ is the degree.\\
    Examples: $y=x^2$, $y=-\frac{1}{2}x^3$

\item {\bf Polynomials}: sum of monomials.\\
    Examples: $y=-\frac{1}{2}x^3+x^2$, $y=3x+5$\\
    The degree of a polynomial is the highest degree of its monomial terms.  Also, it's often a good idea to write polynomials with terms in decreasing degree.

\item {\bf Rational Functions}: ratio of two polynomials.\\
    Examples: $y=\frac{x}{2}$, $y=\frac{x^2+1}{x^2-2x+1}$

\item {\bf Exponential Functions}: Example: $y=2^x$

\item {\bf Trigonometric Functions}: Examples: $y=\cos(x)$, $y=3\sin(4x)$

\item \parbox[c]{4.75in}{{\bf Linear}: polynomial of degree 1.\\
    <!-- Example: $y=m x + b$, where $m$ is the slope and $b$ is the $y$-intercept.}\epsfxsize=1in \parbox{1in}{\, {\includegraphics[width=.9in, angle = 270]{linear.eps}}} -->

\item \parbox[c]{4.75in}{{\bf Nonlinear}: anything that isn't constant or polynomial of degree 1.\\
    Examples:  $y=x^2+2x+1$, $y=\sin(x)$, $y=\ln(x)$, $y=e^x$}
    <!-- \parbox{1in}{\,  {\includegraphics[width=.9in, angle = 270]{nonlin.eps}}} -->
\item[]
\end{verbatim}

\textbackslash{}end\{itemize\}

\section{Log, Ln, and e}\label{log-ln-and-e}

\textbf{Relationship of logarithmic and exponential functions}:
\[y=\log_a(x) \iff a^y=x\]

The log function can be thought of as an inverse for exponential
functions. \(a\) is referred to as the ``base" of the logarithm.

\textbf{Common Bases}: The two most common logarithms are base 10 and
base \(e\).

\begin{enumerate}
\def\labelenumi{\arabic{enumi}.}
\tightlist
\item
  Base 10: \(\quad y=\log_{10}(x) \iff 10^y=x\). The base 10 logarithm
  is often simply written as ``\(\log(x)\)" with no base denoted.
\item
  Base \(e\): \(\quad y=\log_e(x) \iff e^y=x\). The base \(e\) logarithm
  is referred to as the
  \texttt{natural"\ logarithm\ and\ is\ written\ as}\(\ln(x)\)``.
\end{enumerate}

\begin{comment}
            {\includegraphics[width=1in, angle = 270]{ln.eps}} \,  {\includegraphics[width=1in, angle = 270]{exp.eps}}
            \end{comment}

\textbf{Properties of exponential functions:}

\begin{itemize}
\tightlist
\item
  \(a^x a^y = a^{x+y}\)
\item
  \(a^{-x} = 1/a^x\)
\item
  \(a^x/a^y = a^{x-y}\)
\item
  \((a^x)^y = a^{x y}\)
\item
  \(a^0 = 1\)
\end{itemize}

\textbf{Properties of logarithmic functions} (any base):

Generally, when statisticians or social scientists write \(\log(x)\)
they mean \(\log_e(x)\). In other words:
\(\log_e(x) \equiv \ln(x) \equiv \log(x)\)

\[\log_a(a^x)=x\] and \[a^{\log_a(x)}=x\]

\begin{itemize}
\tightlist
\item
  \(\log(x y)=\log(x)+\log(y)\)
\item
  \(\log(x^y)=y\log(x)\)
\item
  \(\log(1/x)=\log(x^{-1})=-\log(x)\)
\item
  \(\log(x/y)=\log(x\cdot y^{-1})=\log(x)+\log(y^{-1})=\log(x)-\log(y)\)
\item
  \(\log(1)=\log(e^0)=0\)
\end{itemize}

\textbf{Change of Base Formula}: Use the change of base formula to
switch bases as necessary: \[\log_b(x) = \frac{\log_a(x)}{\log_a(b)}\]

Example: \[\log_{10}(x) = \frac{\ln(x)}{\ln(10)}\]

\begin{itemize}
\item $\log_{10}(\sqrt{10})=\log_{10}(10^{1/2}) = $
\item $\log_{10}(1)=\log_{10}(10^{0}) = $
\item $\log_{10}(10)=\log_{10}(10^{1}) = $
\item $\log_{10}(100)=\log_{10}(10^{2}) = $
\item $\ln(1)=\ln(e^{0}) = $
\item $\ln(e)=\ln(e^{1}) = $
\end{itemize}

\section{Other Useful Functions}\label{other-useful-functions}

\textbf{Factorials!:}

\[x! = x\cdot (x-1) \cdot (x-2) \cdots (1)\]

\textbf{Modulo:} Tells you the remainder when you divide one number by
another. Can be extremely useful for programming.

\begin{itemize}
\tightlist
\item
  \texttt{x mod y} or \texttt{x \% y}
\item
  \(17 \mod 3 = 2\)
\item
  \(100 \ \% \ 30 = 10\)
\end{itemize}

\textbf{Summation:}

\[\sum\limits_{i=1}^n x_i = x_1+x_2+x_3+\cdots+x_n\]

\begin{itemize}
\item $\sum\limits_{i=1}^n c x_i = c \sum\limits_{i=1}^n x_i $
\item $\sum\limits_{i=1}^n (x_i + y_i) =  \sum\limits_{i=1}^n x_i + \sum\limits_{i=1}^n y_i $
\item $\sum\limits_{i=1}^n c = n c $
\end{itemize}

\textbf{Product:}

\[\prod\limits_{i=1}^n x_i = x_1 x_2 x_3 \cdots x_n\]

Properties:

\begin{itemize}
\item $\prod\limits_{i=1}^n c x_i = c^n \prod\limits_{i=1}^n x_i $
\item $\prod\limits_{i=1}^n (x_i + y_i) =$ a total mess
\item $\prod\limits_{i=1}^n c = c^n $
\end{itemize}

You can use logs to go between sum and product notation. This will be
particularly important when you're learning maximum likelihood
estimation.

\begin{verbatim}
    \begin{eqnarray*}
        \log \bigg(\prod\limits_{i=1}^n x_i \bigg) &=& \log(x_1 \cdot x_2 \cdot x_3 \cdots \cdot x_n)\\
        &=& \log(x_1) + \log(x_2) + \log(x_3) + \cdots + \log(x_n)\\
        &=& \sum\limits_{i=1}^n \log (x_i)
    \end{eqnarray*}
    
    Therefore, you can see that the log of a product is equal to the sum of the logs. We can write this more generally by adding in a constant, $c$:
    
    \begin{eqnarray*}
        \log \bigg(\prod\limits_{i=1}^n c x_i\bigg) &=& \log(cx_1 \cdot cx_2 \cdots cx_n)\\
        &=& \log(c^n \cdot x_1 \cdot x_2 \cdots x_n)\\
        &=& \log(c^n) + \log(x_1) + \log(x_2) + \cdots + \log(x_n)\\\\
        &=& n \log(c) +  \sum\limits_{i=1}^n \log (x_i)\\
    \end{eqnarray*} 
\end{verbatim}

\section{Graphing Functions}\label{graphing-functions}

What can a graph tell you about a function?

\begin{enumerate}
        \item Is the function increasing or decreasing?  Over what part of the domain?
        \item How ``fast" does it increase or decrease?
        \item Are there global or local maxima and minima?  Where?
        \item Are there inflection points?
        \item Is the function continuous?
        \item Is the function differentiable?
        \item Does the function tend to some limit?
        \item Other questions related to the substance of the problem at hand.
    \end{enumerate}

\section{Solving for Variables and Finding
Inverses}\label{solving-for-variables-and-finding-inverses}

Sometimes we're given a function \(y=f(x)\) and we want to find how
\(x\) varies as a function of \(y\). If \(f\) is a one-to-one mapping,
then it has an inverse.

Use algebra to move \(x\) to the left hand side (LHS) of the equation
and so that the right hand side (RHS) is only a function of
\(y\).\textbackslash{}

Examples: (we want to solve for \(x\))

\begin{enumerate}
            \item $y=3x+2 \quad\Longrightarrow\quad -3x=2-y \quad\Longrightarrow\quad 3x=y-2 \quad\Longrightarrow\quad x=\frac{1}{3}(y-2)$
            \item $y=3x-4z+2 \quad \Longrightarrow\quad y+4z-2=3x \quad\Longrightarrow\quad x=\frac{1}{3}(y+4z-2)$
            \item $y=e^x+4 \quad\Longrightarrow\quad y-4=e^x \quad\Longrightarrow\quad \ln(y-4)=\ln(e^x)\quad\Longrightarrow\quad x=\ln(y-4)$
\end{enumerate}

Sometimes (often?) the inverse does not exist. For example, we're given
the function \(y=x^2\) (a parabola). Solving for \(x\), we get
\(x=\pm\sqrt{y}\). For each value of \(y\), there are two values of
\(x\).

\section{Finding the Roots or Zeroes of a
Function}\label{finding-the-roots-or-zeroes-of-a-function}

Solving for variables is especially important when we want to find the
\{\bf roots\} of an equation: those values of variables that cause an
equation to equal zero. Especially important in finding equilibria and
in doing maximum likelihood estimation.

Procedure: Given \(y=f(x)\), set \(f(x)=0\). Solve for \(x\).

Multiple Roots:
\[f(x)=x^2 - 9 \quad\Longrightarrow\quad 0=x^2 - 9 \quad\Longrightarrow\quad 9=x^2 \quad\Longrightarrow\quad \pm \sqrt{9}=\sqrt{x^2} \quad\Longrightarrow\quad \pm 3=x\]

\textbf{Quadratic Formula:} For quadratic equations \(ax^2+bx+c=0\), use
the quadratic formula: \[x=\frac{-b\pm\sqrt{b^2-4ac}}{2a}\]\}

Examples:

\begin{enumerate}
        \item $f(x)=3x+2$ \\
        % $\quad\Longrightarrow\quad 3x+2=0 
        % \quad\Longrightarrow\quad x=-\frac{2}{3}$\\
        \item $f(x)=e^{-x}-10$ \\
        %$\quad\Longrightarrow\quad e^{-x}-10=0 
        %\quad\Longrightarrow\quad e^{-x}=10 \quad\Longrightarrow\quad 
        %x=-\ln(10)$\\
        \item $f(x)=x^2+3x-4=0$  \\
        %$\quad\Longrightarrow\quad x=\{1,-4\}$\\
\end{enumerate}

\section{The Limit of a Function}\label{the-limit-of-a-function}

\begin{verbatim}
We're often interested in determining if a function $f$ approaches some number $L$ as its independent variable $x$ moves to some number $c$ (usually 0 or $\pm\infty$).  If it does, we say that the limit of $f(x)$, as $x$ approaches $c$, is $L$: $\lim\limits_{x \to c} f(x)=L$.
\end{verbatim}

For a limit \(L\) to exist, the function \(f(x)\) must approach \(L\)
from both the left and right.

\textbf{Limit of a function}. Let \(f(x)\) be defined at each point in
some open interval containing the point \(c\). Then \(L\) equals
\(\lim\limits_{x \to c} f(x)\) if for any (small positive) number
\(\epsilon\), there exists a corresponding number \(\delta>0\) such that
if \(0<|x-c|<\delta\), then \(|f(x)-L|<\epsilon\).

Note: f(x) does not necessarily have to be defined at \(c\) for
\(\lim\limits_{x \to c}\) to exist.

\textbf{Uniqueness}: \(\lim\limits_{x \to c} f(x)=L\) and
\(\lim\limits_{x \to c} f(x)=M \Longrightarrow L=M\)

Properties: Let \(f\) and \(g\) be functions with
\(\lim\limits_{x \to c} f(x)=A\) and \(\lim\limits_{x \to c} g(x)=B\).

\begin{enumerate}
        \item $\lim\limits_{x \to c}[f(x)+g(x)]=\lim\limits_{x \to c} f(x)+ \lim\limits_{x \to c} g(x)$
        \item $\lim\limits_{x \to c} \alpha f(x) = \alpha \lim\limits_{x \to c} f(x)$
        \item $\lim\limits_{x \to c} f(x) g(x) = [\lim\limits_{x \to c} f(x)][\lim\limits_{x \to c} g(x)]$
        \item $\lim\limits_{x \to c} \frac{f(x)}{g(x)} = \frac{\lim\limits_{x \to c} f(x)}{\lim\limits_{x \to c} g(x)}$, provided $\lim\limits_{x \to c} g(x)\ne 0$.
            \item Note: In a couple days we'll talk about L'H\^opital's Rule, which uses simple calculus to help find the limits of functions like this.
\end{enumerate}

Examples:

\begin{enumerate}
\item $\lim\limits_{x \to c} k =$
\item $\lim\limits_{x \to c} x =$

\begin{comment}
        \item \parbox[t]{3.75in}{$\lim\limits_{x\to 0} |x| =$} \parbox[t]{1in}{\,{\includegraphics[width=1in, angle = 270]{abs.eps}}} % = 0
        \item \parbox[t]{3.75in}{$\lim\limits_{x\to 0} \left(1+\frac{1}{x^2}\right)=$} \parbox[t]{1in}{\,  {\includegraphics[width=1in, angle = 270]{1p1ovrx2.eps}}} %\infty
\end{comment}
        
\item $\lim\limits_{x\to 2} (2x-3) =$ %= 2\lim\limits_{x\to 2} x- 3\lim\limits_{x\to 2} 1 = 2\times 2 - 3\times 1 = 1 
\item $\lim\limits_{x \to c} x^n = $ %[\lim\limits_{x \to c} x]\cdots[\lim\limits_{x \to c} x] = c\cdots c =c^n 
\end{enumerate}

Types of limits:

\begin{enumerate}
\def\labelenumi{\arabic{enumi}.}
\tightlist
\item
  Right-hand limit: The value approached by \(f(x)\) when you move from
  right to left.

  \begin{comment}
      \parbox[t]{2in}{Example:  $\lim\limits_{x\to 0^+} \sqrt{x} = 0$}\parbox[t]{1in}{\,  {\includegraphics[width=1in, angle = 270]{sqrt.eps}}}
      \end{comment}
\item
  Left-hand limit: The value approached by \(f(x)\) when you move from
  left to right.\textbackslash{}
\item
  Infinity: The value approached by \(f(x)\) as x grows infinitely
  large. Sometimes this may be a number; sometimes it might be
  \(\infty\) or \(-\infty\).\textbackslash{}
\item
  Negative infinity: The value approached by \(f(x)\) as x grows
  infinitely negative. Sometimes this may be a number; sometimes it
  might be \(\infty\) or \(-\infty\).\textbackslash{}
\end{enumerate}

\begin{comment}
\parbox[t]{2in}{Example:  $\lim\limits_{x\to \infty} 1/x = \lim\limits_{x\to -\infty} 1/x= 0$}\parbox[t]{1in}{\,  {\includegraphics[width=1in, angle = 270]{1ovrx.eps}}}\\ 
\end{comment}

\section{Continuity}\label{continuity}

\textbf{Continuity}: Suppose that the domain of the function \(f\)
includes an open interval containing the point \(c\). Then \(f\) is
continuous at \(c\) if \(\lim\limits_{x \to c} f(x)\) exists and if
\(\lim\limits_{x \to c} f(x)=f(c)\). Further, \(f\) is continuous on an
open interval \((a,b)\) if it is continuous at each point in the
interval.

\begin{itemize}
\tightlist
\item
  Examples: Continuous functions.
\end{itemize}

\begin{comment}
    \parbox[t]{1.5in}{\hfill$f(x)=\sqrt{x}\quad$}\parbox[t]{1in}{\,  {\includegraphics[width=1in, angle = 270]{sqrt.eps}}}
    \parbox[t]{1.5in}{\hfill$f(x)=e^x\quad$}\parbox[t]{1in}{\,  {\includegraphics[width=1in, angle = 270]{exp.eps}}}
    \item[] Examples: Discontinuous functions.\\
    \parbox[t]{1.5in}{\hfill$f(x)=\mbox{floor}(x)\quad$}\parbox[t]{1in}{\,  {\includegraphics[width=1in, angle = 270]{floor.eps}}}
    \parbox[t]{1.5in}{\hfill$f(x)=1+\frac{1}{x^2}\quad$}\parbox[t]{1in}{\,  {\includegraphics[width=1in, angle = 270]{1p1ovrx2.eps}}}
\end{comment}

Properties:

\begin{enumerate}
\def\labelenumi{\arabic{enumi}.}
\tightlist
\item
  If \(f\) and \(g\) are continuous at point \(c\), then \(f+g\),
  \(f-g\), \(f \cdot g\), \(|f|\), and \(\alpha f\) are continuous at
  point \(c\) also. \(f/g\) is continuous, provided \(g(c)\ne 0\).
\item
  Boundedness: If \(f\) is continuous on the closed bounded interval
  \([a,b]\), then there is a number \(K\) such that \(|f(x)|\le K\) for
  each \(x\) in \([a,b]\).
\item
  Max/Min: If \(f\) is continuous on the closed bounded interval
  \([a,b]\), then \(f\) has a maximum and a minimum on \([a,b]\). They
  may be located at the end points.
\end{enumerate}

\section{Sets}\label{sets}

\textbf{Interior Point}: The point \(\bf x\) is an interior point of the
set \(S\) if \(\bf x\) is in \(S\) and if there is some
\(\epsilon\)-ball around \(\bf x\) that contains only points in \(S\).
The \{\bf interior\} of \(S\) is the collection of all interior points
in \(S\). The interior can also be defined as the union of all open sets
in \(S\).

\begin{itemize}
\tightlist
\item
  If the set \(S\) is circular, the interior points are everything
  inside of the circle, but not on the circle's rim.
\item
  Example: The interior of the set \(\{ (x,y) : x^2+y^2\le 4 \}\) is
  \(\{ (x,y) : x^2+y^2< 4 \}\) .
\end{itemize}

\textbf{Boundary Point}: The point \(\bf x\) is a boundary point of the
set \(S\) if every \(\epsilon\)-ball around \(\bf x\) contains both
points that are in \(S\) and points that are outside \(S\). The
\{\bf boundary\} is the collection of all boundary points.

\begin{itemize}
\tightlist
\item
  If the set \(S\) is circular, the boundary points are everything on
  the circle's rim.
\item
  Example: The boundary of \(\{ (x,y) : x^2+y^2\le 4 \}\) is
  \(\{ (x,y) : x^2+y^2 = 4 \}\).
\end{itemize}

\textbf{Open}: A set \(S\) is open if for each point \(\bf x\) in \(S\),
there exists an open \(\epsilon\)-ball around \(\bf x\) completely
contained in \(S\).

\begin{itemize}
\tightlist
\item
  If the set \(S\) is circular and open, the points contained within the
  set get infinitely close to the circle's rim, but do not touch it.
\item
  Example: \(\{ (x,y) : x^2+y^2<4 \}\)
\end{itemize}

\textbf{Closed}: A set \(S\) is closed if it contains all of its
boundary points. * If the set \(S\) is circular and closed, the set
contains all points within the rim as well as the rim itself. * Example:
\(\{ (x,y) : x^2+y^2\le 4 \}\) * Note: a set may be neither open nor
closed. Example: \(\{ (x,y) : 2 < x^2+y^2\le 4 \}\)

\textbf{Complement}: The complement of set \(S\) is everything outside
of \(S\).\textbackslash{} * If the set \(S\) is circular, the complement
of \(S\) is everything outside of the circle. * Example: The complement
of \(\{ (x,y) : x^2+y^2\le 4 \}\) is \(\{ (x,y) : x^2+y^2 > 4 \}\).

\textbf{Closure}: The closure of set \(S\) is the smallest closed set
that contains \(S\). Example: The closure of \(\{ (x,y) : x^2+y^2<4 \}\)
is \(\{ (x,y) : x^2+y^2\le 4 \}\)

\textbf{Bounded}: A set \(S\) is bounded if it can be contained within
an \(\epsilon\)-ball.

\begin{itemize}
\tightlist
\item
  Examples: Bounded: any interval that doesn't have \(\infty\) or
  \(-\infty\) as endpoints; any disk in a plane with finite radius.
\item
  Unbounded: the set of integers in \({\bf R}^1\); any ray.
\end{itemize}

\textbf{Compact}: A set is compact if and only if it is both closed and
bounded.

\textbf{Empty}: The empty (or null) set is a unique set that has no
elements, denoted by \{\} or \o.
\footnote{The set, $S$, denoted by \{\o \} is technically \emph{not} empty. That is because this set contains the empty set within it, so $S$ is not empty.}

\begin{itemize}
\tightlist
\item
  Examples: The set of squares with 5 sides; the set of countries south
  of the South Pole.\textbackslash{}
\end{itemize}

\chapter{Methods}\label{methods}

We describe our methods in this chapter.

\chapter{Applications}\label{applications}

Some \emph{significant} applications are demonstrated in this chapter.

\section{Example one}\label{example-one}

\section{Example two}\label{example-two}

\chapter{Final Words}\label{final-words}

We have finished a nice book.

\bibliography{book.bib,packages.bib}


\end{document}
